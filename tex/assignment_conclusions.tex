\section{Conclusion}

In building a resilient cyber-security organisation, it is not enough to focus on the technical details of the protections that
have been put in place.  The CISO must assume that their organisation will experience a cyber-attack and should be able to discuss
the operational and reputation risks that the organisation will face with the board and senior executives.

We looked at one specific style of attack, process injection, and saw how the evolving set of techniques means that the
organisation's defence is always a shifting game of cat-and-mouse; detection and protection are not always guaranteed.

Looking at this specific new process injection, designed to evade typical end-point defence systems, we built a case study that
also took into account the motivations of a threat actor known to exploit such a vulnerability --- the LockBit malware group ---
to ask how can  use the knowledge of this new technique to strengthen both our defences \textit{and} our planning for our response if ever an
attack is successful.

While the most obvious mitigants to a cyber-event are to engage with vendors on new attack vectors, update and patch existing systems and
restrict the attack surface as much as possible, the CISO must translate these technical responses into business level questions on operational
resilience.

Building board-level trust in cyber-security defences is the foundation on which a CISO can build a successful cyber-security team.  And
building those foundations requires framing the communication of cyber-security issues, not in technical terms, but in the robustness
of the fundamental services that the organisation must deliver every working day.


%Such internal case studies are just one way that a cyber

%The outcome of such reviews should feed into 
%\begin{itemize}
%\item Recap of key points related to malware attacks and process injection on Windows
%\item Call to action for cybersecurity professionals and organizations
%\item Ongoing vigilance and adaptation to the evolving threat landscape
%\item Recap of the challenges identified in the Harvard Business Review article.
%\item Summary of the proposed strategic enhancement for CISOs.
%\item Emphasis on the importance of elevating the CISO's role for improved cybersecurity.
%\end{itemize}

%Communicating cyber and information security risks and mitigants to senior executives and board members is an important primary objective of CISOs.
%By regularly giving honest assesments of their organisations defences and weaknesses, the CISO can improve trust, explain budget requirements, and work to a convergence of opinion within the
%leadership and givernance teams..

%Part of this communication should be regular reporting developed by the information security group that can be used to generate board presentations and executive summaries for reporting on the
%organisation's preparedness.

%The reports deliverss an example an on in-depth review of such an immergent threat; process injection techniques, EDR capabilities, and weaknesses in that defence.  The outcome of that review are
%a number risks with-in the cyber-security system and  steps that the information security group can perform to mitigate those risks.

%The identified risks are:
%\begin{itemize}
%\item Attacks can be from both zero-day exploits of systems, such as secure remote access gateways such as Netscaler as well as delivery through email phishing attacks.
%\item The capabilities of EDRs from different vendors are diffent and may have false negatives for different types of attacks.
%\item \ldots \textit{add in good cyber-security policies}
%\end{itemize}

%Mitigants that the CISO can put into place are:
%\begin{itemize}
%\item Ensure that the organisation has a development plan for security professionals within the orgainisation, and that these people develop skills to test and appraise systems \ldots \textit{red team blue teams, pen test, etc}.
%\item  All systems, not just front line defences, such as anti-virus and EDRs, should have procedures for timely and robust patching and upgrades.
%\item The CISO should ensure that vendors are challenged to  validate their systems for new attacks and regularly test systems capabilities. 
%\end{itemize}