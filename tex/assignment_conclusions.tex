\section{Conclusions}

Communicating cyber and information security risks and mitigants to senior executives and board members is an important primary objective of CISOs.
By regularly giving honest assesments of their organisations defences and weaknesses, the CISO can improve trust, explain budget requirements, and work to a convergence of opinion within the
leadership and givernance teams..

Part of this communication should be regular reporting developed by the information security group that can be used to generate board presentations and executive summaries for reporting on the
organisation's preparedness.

The reports deliverss an example an on in-depth review of such an immergent threat; process injection techniques, EDR capabilities, and weaknesses in that defence.  The outcome of that review are
a number risks with-in the cyber-security system and  steps that the information security group can perform to mitigate those risks.

The identified risks are:
\begin{itemize}
\item Attacks can be from both zero-day exploits of systems, such as secure remote access gateways such as Netscaler as well as delivery through email phishing attacks.
\item The capabilities of EDRs from different vendors are diffent and may have false negatives for different types of attacks.
\item \ldots \textit{add in good cyber-security policies}
\end{itemize}

Mitigants that the CISO can put into place are:
\begin{itemize}
\item Ensure that the organisation has a development plan for security professionals within the orgainisation, and that these people develop skills to test and appraise systems \ldots \textit{red team blue teams, pen test, etc}.
\item  All systems, not just front line defences, such as anti-virus and EDRs, should have procedures for timely and robust patching and upgrades.
\item The CISO should ensure that vendors are challenged to  validate their systems for new attacks and regularly test systems capabilities. 
\end{itemize}