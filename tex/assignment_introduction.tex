\section{Introduction}

Host-based code injection attack (HBCIA) techniques (\cite{6999410}) continue to evolve since a method to map an external module into a target process by Microsoft (\cite{Ghizzoni:2004}) was published, to support applications such as ``anti-virus programs, profilers, debuggers, and pseudo-localization testing'', and quickly used by malicious actors to insert DLLs into target processes (\cite{4420399}).

A cat-and-mouse game beween security groups on the one hand, and the evasion and increased sophistication and novelty on the part of malicious actors on the other.  \textit{need to introduct reinforcement techniques used by endpoint security.} 

Recently a new host-base code injection (HBCI) method targeting the \textbf{RWX} section \textit{(include more detail here)} was identified by researchers (\cite{Peixoto:2023}).  This report is a review of this technique and an overview of this class of attacks.

Section 2 is a literature review of HBCI methods and endpoint security that is typically relied upon to prevent these types of attack.  We will then look at methods a identifying these attacks and look at the likelihood of Endpoint Security products of identifying the attack.

Section 3 an investigation into the Mockingjay attack and against a recently published paper ``Procguard'' \cite{10063560} and asks weather this attack method would lickly be caught.

% {jwang,mcj123,ZiangLi,2018302180148,iwangjye}@whu.edu.cn

Section 4 is an implementation of the attack and will look at manually identifying an infected process and generating artifacts that could be used in automating the process.

Section 5 is an evaluation section, reflecting on the project.

Section 7 is the conclusion and will suggest further research that could be undertaken to use reinforcement to identify ``RWX'' code injection.
