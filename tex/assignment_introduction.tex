\section{Introduction}

% how does this assignment relate to cyber-security

It only takes a a few minutes for a new exploit to be found to get access to a companies computing assets \textit{find and cite the recent citrix bug that allowed an attacker to kick a user of their session and gain access}.

From social engineering a user to click on a nefarious link \textit{find an example exploit} or opening a spreadsheet \textit{put in example of process injection in an Excel macro}, there's rich pickings for an attacker to target a system.

For organisations with assets that threat actors want to attack, \href{https://www.cisa.gov/news-events/cybersecurity-advisories/aa23-165a}{understand these groups and their techniques} and \href{https://www.cshub.com/attacks/articles/five-active-ransomware-gangs-and-their-tactics-part-one}{tactics} are a vital part of an organisation's cyber-security education and defence.

A cat-and-mouse game beween security groups on the one hand, and the evasion and increased sophistication and novelty on the part of malicious actors on the other.  \textit{need to introduce reinforcement techniques used by endpoint security.} 

For vendors of XDR solutions investigating existing and emerging threats is an on-going process to a\href{https://research.tue.nl/files/305661196/Olteanu_I.C..pdf}{evaluating the response effectiveness of their XDR technology}.

Terminology \ldots Host-based code injection attack (HBCIA) techniques \autocite{Barabosch:2014} continue to evolve since a method to map an external module into a target process by Microsoft \autocite{Ghizzoni:2004} was published, to support applications such as ``anti-virus programs, profilers, debuggers, and pseudo-localization testing'', and quickly used by malicious actors to insert DLLs into target processes \autocite{Jang:2007}.

Recently a new host-base code injection (HBCI) method targeting the code sections with Read Write eXecute (RWX)  \textit{(include more detail here)} was identified by researchers \autocite{Peixoto:2023}.  This report is a review of this technique and an overview of this class of attacks.

This paper reports on a new process injection technique \autocite{Peixoto:2023} in relation to the protections offered by EDR and XDR technologies, and specifically with behaviours that could be used to detect API attacks \autocite{Wang:2022}.  We will analyse the ``Mockingjay'' attack and identify the defences offered by modern XDR systems and ask if there's a credible chance of evading detection.  By looking at an implementation of the attack we will ask in what ways a threat detection system strengthen it's defences, and what artifacts could be automatically produced that could demonstrate any anomoly in a systems behaviour. 

Section 2 is a literature review of HBCI methods and endpoint security that is typically relied upon to prevent these types of attack.  We will then look at methods a identifying these attacks and look at the likelihood of Endpoint Security products of identifying the attack.

Section 3 an investigation into the Mockingjay attack and against a recently published paper ``Procguard'' \autocite{Wang:2022} and asks weather this attack method would lickly be caught.

% {jwang,mcj123,ZiangLi,2018302180148,iwangjye}@whu.edu.cn

Section 4 is an implementation of the attack and will look at manually identifying an infected process and generating artifacts that could be used in automating the process.

Section 5 is an evaluation section, reflecting on the project.

Section 6 is the conclusion and will suggest further research that could be undertaken to use reinforcement to identify ``RWX'' code injection.
