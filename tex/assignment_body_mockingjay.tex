\section{Case Study: Introducing MockingJay; New Process Injection Attacks}

\href{https://www.securityjoes.com/post/process-mockingjay-echoing-rwx-in-userland-to-achieve-code-execution}{Mockingjay echoing in userland to achieve code execution}

\href{https://www.linkedin.com/posts/john-stigerwalt-90a9b4110_mockingjay-memory-allocation-primitive-activity-7083050050158743552-Hgyw}{mockingjay memory allocation primative}

\href{https://whiteknightlabs.com/2023/07/06/mockingjay-memory-allocation-primitive/}{white knights mocking jay}

%\subsection{Real-world examples of malware employing process injection on Windows}
\subsection{Detailed analysis of new process injection attacks.}

\subsection{Examination of how these attacks could potentially evade an organization's EDR system.}

%\subsection{Analysis of the impact and consequences of these incidents}
\subsection{Significance of the use case in assessing and improving the organization's cybersecurity measures.}

\textbf{\textcite{Inam:2023}: Provenance-based system auditing and how it could improve EDR malware detection}

Provenance-based system auditing refers to the capture and analysis of detailed logs that track the history and dependencies between system events and entities. This allows reconstructing the chain of events leading up to an attack (backtracing) or the ramifications of an attack (forward tracing).

Provenance auditing builds contextual graphs of system execution that can detect complex malware behaviors often missed by traditional EDR monitoring. This improves detection rates and investigation efficiency.
