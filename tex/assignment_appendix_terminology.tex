\section*{Appendix: Terminology and Abbreviations}

\begin{multicols}{2}

  \entry{ADC}{Application delivery controller: purpose-built networking appliance used to improve the performance, security, and resiliency of applications delivered over the web.}
  
  \entry{ASLR}{Address space layout randomization is a technique to guard against buffer-overflow attacks.}

  \entry{CISA}{Cybersecurity and Infrastructure Security Agency: US agency \ldots}

  \entry{CSF}{NIST Cybersecurity Framework is a set of practices to help organisations improve the cybersecurity culture.}
  
  \entry{ICE}{In-memory Code Execution attacks only execute malicious operations in memory and leaving little evidence on disk.}

  \entry{NIST}{National Institute fof Standards and Technology; part of the US Department of Commerce.}
  
  \entry{PsExec}{PsExec is not malware itself, but it can be used by malware and attackers to perform malicious actions. PsExec is a legitimate tool that allows users to run programs on remote systems.}
  
  \entry{SIEM}{Security information and event management: combining security information management (SIM) and event mangement (SEM), these systems enable organizations detect, analyze, and respond to security threats.  This is achieved by collecting event log data from many sources to identify anomolous activities deviating from the norm with real-time analysis.  These systems seek to take appropriate action before any threat harms business operations.
    security threats before they harm business operations.}

  \entry{SOC}{Security operations center.}

  \entry{RWX}{Read-Write-eXecute code sections in \ldots}
  
  \entry{TTP}{Tactics, Techniques, and Procedures is a key concept in cybersecurity and threat intelligence. The purpose is to identify patterns of behavior which can be used to defend against specific strategies and threat vectors used by malicious actors.}
  
\end{multicols}