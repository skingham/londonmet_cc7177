\documentclass{article}

\usepackage[english]{babel}
\usepackage[a4paper,top=2cm,bottom=2cm,left=3cm,right=3cm,marginparwidth=1.75cm]{geometry}

% Useful packages
\usepackage[backend=biber, sorting=nyt, style=authoryear-ibid]{biblatex}
\usepackage[outputdir=pdf]{minted}
\usepackage{graphicx}
\graphicspath{ {figures/} }
\usepackage[colorlinks=true, allcolors=blue]{hyperref}
\usepackage{parskip}
%\usepackage{lipsum}
\usepackage{setspace}
\doublespacing
\usepackage{multicol}
\usepackage{multirow}
\usepackage{fontspec}
\usepackage{fontawesome5}
\usepackage{amsmath}
\usepackage{csquotes}

% Document style
%\usepackage[utf8]{inputenc} % Required for inputting international characters
%\usepackage[T1]{fontenc} % Output font encoding for international characters
%\usepackage{palatino} % Use the Palatino font
%\usepackage{microtype} % Improves spacing
%\usepackage[bf,sf,center]{titlesec} % Required for modifying section titles - bold, sans-serif, centered
%\usepackage{fancyhdr} % Required for modifying headers and footers
%\fancyhead[L]{\textsf{\rightmark}} % Top left header
%\fancyhead[R]{\textsf{\leftmark}} % Top right header
%\renewcommand{\headrulewidth}{1.4pt} % Rule under the header
%\fancyfoot[C]{\textbf{\textsf{\thepage}}} % Bottom center footer
%\renewcommand{\footrulewidth}{1.4pt} % Rule under the footer
%\pagestyle{fancy} % Use the custom headers and footers throughout the document

% Appendix dictionary template: print each word on the page
% \markboth{}{} prints the first word on the page in the top left header and the last word in the top right
%\newcommand{\entry}[4]{\markboth{#1}{#1}\textbf{#1}\ {(#2)}\ \textit{#3}\ $\bullet$\ {#4}}
\newcommand{\entry}[2]{\markboth{#1}{#1}\textbf{#1}\ $\bullet$\ {#2}}


%----------------------------------------------------------------------------------------
%
% Title Page: clear and concise
% e.g. technical report title and module code and title, plus your name and student ID number
%
% Abstract Page
%
% Contents Page {shows structure of report - section numbers, heading andpages}
%
% Introduction {very brief description of
%   aims (general) and
%   objectives (what is done to achieve the aims to put report within context/sets the scene for thereader (e.g. where does this development fit within the field);
%               what are theproblems/issues of the subject area}
%
% Body of report {main part of the report; could be divided in several sectionspending the research/discussions you undertake}
%
% Evaluation {evaluate the results; can be a subsection of the Body of report}
%
% Conclusions {condensed version of body; briefly gives key findings andfuture works}
%
% References (Bibliography) {demonstration of your referencing skills}
%
% Appendices {optional, if there is any}
%
%----------------------------------------------------------------------------------------

%\bibliography{references}
\addbibresource{references.bib}

\title{Can Mockingjay Evade your CISO: A Review of Current Process Injection Attacks}
\author{Stuart Kingham: ID 21014912}

\begin{document}
\doublespacing

%\begin{center}
%{\LARGE Current and emerging online cyber-attacks, threats, and criminal acts, including related digital crimes}
%\end{center} 
%\vspace*{\fill}


%\maketitle
\begin{titlepage}
  \topskip0pt
  \vspace*{\fill}
  \begin{center}
       \vspace*{1cm}

       {\LARGE Current and emerging online cyber-attacks, threats, and criminal acts, including related digital crimes}

       \vspace*{1cm}
       {\large \textbf{ Can Mockingjay Evade your CISO}}
       
       \vspace{0.2cm}
       {\large A Review of Current Process Injection Attacks}
            
       %\vspace{1.5cm}

       \vfill

       \textbf{Stuart Kingham: ID 2101491}

       \vfill
                        
       \vspace{0.8cm}
     
       %\includegraphics[width=0.4\textwidth]{university}

       CC7177: Cybercrime and Cyber-Security\\
       School of Computing and Digital Media\\
       London Metropolitan University\\
       December 11, 2023
            
  \end{center}
  \vspace*{\fill}
\end{titlepage}

\pagebreak

\begin{abstract}
  CISOs and boards are not communicating well, with divergent views on the risks of cyber-attacks.
  Instead of approaching cyber-security as a technical issue that requires specific expertise in information technology,
  boards should be focusing on the strategic imperative of cyber-security.
  They should be engaging with their CISO to discuss and understand the organisational risks that cyber-attacks expose the organisation to.
  By refocusing on resilience, CISOs can engage more meaningfully with board members.
  In this report we investigate a new Windows process injection attack. In understand the threat we propose a set of
  recommendations a CISO would expect to mitigate the risk of a successful attack.
  While doing this we show how the exercise can also generate recommendations that improve operational resilience as well as the protective mitigants.
  In this way the CISO can frame cyber-defence planning as tackling specific strategic threats to an organisation. 
  By fully understanding possible attacks and threat actors, the CISO can demonstrate the safeguards to the organisation's services and capabilities.
  In creating a more resilient cyber-security system, the CISO exists as an important strategic partner in the reduction of operational risk.
\end{abstract}

\newpage

\singlespacing 
\tableofcontents
\listoffigures
\listoftables
\doublespacing

\pagenumbering{roman}
\pagenumbering{arabic}
\newpage

\section{Introduction}

Your introduction goes here! Simply start writing your document and use the Recompile button to view the updated PDF preview. Examples of commonly used commands and features are listed below, to help you get started.

Once you're familiar with the editor, you can find various project setting in the Overleaf menu, accessed via the button in the very top left of the editor. To view tutorials, user guides, and further documentation, please visit our \href{https://www.overleaf.com/learn}{help library}, or head to our plans page to \href{https://www.overleaf.com/user/subscription/plans}{choose your plan}.



% Section 2 is a literature review of HBCI methods and endpoint security that is typically relied upon to prevent these types of attack. We will then look at methods a identifying these attacks and look at the likelihood of Endpoint Security products of identifying the attack.

% Outline

% introduce EDRs and how they are used to protect organisations

% introduce process injection attacks: MITRE, and NT process attacks

% how do EDRs

\section{Proess Injection Attack and EDR Repsonse Primer}


Terminology \ldots Host-based code injection attack (HBCIA) techniques \autocite{Barabosch:2014} continue to evolve since a method to map an external module into a target process by Microsoft \autocite{Ghizzoni:2004} was published, to support applications such as ``anti-virus programs, profilers, debuggers, and pseudo-localization testing'', and quickly used by malicious actors to insert DLLs into target processes \autocite{Jang:2007}.

Recently a new host-base code injection (HBCI) method targeting the code sections with Read Write eXecute (RWX)  \textit{(include more detail here)} was identified by researchers \autocite{Peixoto:2023}.  This report is a review of this technique and an overview of this class of attacks.


End-point Detection and Response Systems (EDR) have developed to counter these myriad threats.  These systems are gaining in sophistication and adversaries
are on the hunt for new ways to evade these systems.

In this report we look at a new variation of one attack technique, process injection, and ask how EDRs protect against these attacks and how
this new technique would evade those barriers.  In particular we are looking at the ``Mockingjay'' attach that targets modern Windows machines.

EDRs incorporate protections of increasing levels of sophistication.

At the first level there are static monitoring of specific machine DLLs and processes to detect changes.  Making sure that certain subsystems
have not been modified, had elevated permissions, or unknown processes running on the monitored system.  Signature based

Next, user and entity behaviour analytics (EUBA) tools identify changes in behaviours that could indicate a threat.  This could include the
applications and systems a user normally uses, looking for configuration changes, and scanning log files for changes in behaviour.

Statistical modeling will profile normal activity and continuous monitoring will attempt to descern anomolous behaviour as a threat occurs.

Anomaly detection can be prone to false positives.  The threat actor will look for attacks that could look like inocent behavior.

These protections can be signatu



\subsection{Endpoint Security: EDR}

% \href{https://www.crowdstrike.com/cybersecurity-101/endpoint-security/edr-vs-mdr-vs-xdr/}{EDR vs MDR vs XDR}
HBCI techniques work on injecting code into running processes and having that code executed as part of the normal process
execution.  End-point Detection and Response (EDR) systems \autocite{Hayes:2023} provides realtime visibiluty

Process injection seeks bypass EDR hooks by injecting code into trusted processes with RWX permissions already set.

\subsection{Code Injection Methods}

Talk about Mitre process injection tech T1055 \autocite{Mitre:2017}

\subsubsection{Deploying Payload into Windows Memory Space}

% \href{https://modexp.wordpress.com/2018/07/15/process-injection-sharing-payload/}{Process injection sharing payload}

\autocite{Zhan:2018}

\href{https://www.riskinsight-wavestone.com/en/2023/10/process-injection-using-ntsetinformationprocess/}{PI using NTSetInformationProcess}

\href{https://github.com/elddy/Windows-NTAPI-Injector}{NTAPI injector} 

\href{https://gist.github.com/WKL-Sec/96e17188e4c159c2cdf7ff2c111130cc#file-local-c}{Injector examples in C}

\href{https://www.unknowncheats.me/forum/anti-cheat-bypass/286274-internal-detection-vectors-bypass.html}{internal detection vectors bypass}

\href{https://medium.com/@s12deff/process-injection-with-random-rwx-memory-spaces-3e3651149527}{PI with random RWX memory spaces}


% Section 3 an investigation into the Mockingjay attack and against a recently published paper ``Procguard''  and asks whether this attack method would lickly be caught.
\pagebreak
\section{Case Study: Introducing the MockingJay Process Injection Attacks}

After our review of process injection techniques used by malware groups, we now ask how a CISO in an organisation can stay ahead of the curve,
have a realistic understanding of the threats their organisation might face and develop a resilient cyber-security infrastructure.

Our hypothetical CISO has been active in putting in place an XDR system. This has reduced the manual processes his group needs to
perform --- such as patching individual cybersecurity applications and running separate disjoint processes --- and has leveraged the
vendor's expertise to configure the monitoring of data produced across endpoints, servers, cloud networks and
security information and event management (SEIM) systems.

Our enlightened CISO has empowered his team to research ways to improve cyber-security and looks forward to using their
insights to engage with senior stakeholders within the organisation.  He champions his team's achievements and regularly
briefs the board on improvements to understanding their operational risks and mitigation plans.

One of the team's security engineers has come across a new potential exploit and has produced the following case study for the CISO
and his team.  The Mockingjay attack \autocite{Peixoto:2023} is a real and novel process injection attack created by Security Joes, a
``multi-layered MDR \& incident response company'' based in Tel Aviv.  The Security Joes report is an opportunity to assess the organisation's
defence to process injection attacks and how it would respond to such an attack if successful.

\subsection{Introduction}

\textbf{Problem Statement}:

\begin{enumerate}
\item A new process injection attack has been identified by Securities Joe.  The exploit circumvents the allocation
  and permission APIs that most EDR systems monitor.  It  does this by using existing RWX code sections without invoking
  new threads.
\item We may be vulnerable to this attack which could allow a Windows executable to use self-injection or remote injection to deliver a malware payload into our systems.
\item Recently Citrix has reported vulnerabilities in their NetScaler Application Delivery Controller (ADC) and Gateway products that
  could allow:
  \begin{itemize}
  \item \citetitle{CVE-2023-3467} \autocite{CVE-2023-3467}.
  \item \citetitle{CVE-2023-3519} \autocite{CVE-2023-3519}.
  \item \citetitle{CVE-2023-4966} \autocite{CVE-2023-4966}.
  \end{itemize}
\item The Cybersecurity and Infrastructure Security Agency (CISA) has reported that the malware group LockBit \autocite{CISA:2023a} has
  been found to be actively exploiting CVE-2023-4966 and was able to obtain initial access to systems at Boeing Distribution Inc. \autocite{CISA:2023b}.
\end{enumerate}

Our organisation should be prepared for similar attacks, and we should review our cyber-defences in light of this information.
As always, we should seek to prevent any successful attack, but also improve our cyber-resilience in the face of a successful breach.

This case study will:
\begin{enumerate}
\item Review the salient points of the new attack vector, how it may evade our XDR system and how we may identify an attack.
\item Understand ransomware threat actors such as LockBit and how we can detect and respond to a ransomware group's incursion.
\item Prepare a set of recommendations in line with our Cyber-Security Framework (CSF) \autocite{NIST:2018}.
\end{enumerate}


%\subsection{Real-world examples of malware employing process injection on Windows}
\subsection{Detailed Analysis of New Process Injection Attack}

Attackers could exploit the Mockingjay technique to circumvent EDR/XDR monitoring and anti-virus software by avoiding
common system API calls used by malware.

It does this by searching for windows DLLs with a default RWX memory section that can be exploited to run a malware payload.
This is similar to the Nirvana callback code injection that use NtWriteVirtualMemory and NtProtectVirtualMemory, which are
monitored by EDR/XDRs,  but this method avoids those calls.

The DLL and executables exploited in the report were ``msys-2.0.dll'' and the GNU ``ssh.exe'', which come part of Visual Studio Community.
The security report was able to demonstrate:

\begin{enumerate}
\item Self-injection: build an executable that directly loads the identified DLL, then inject and then execute the payload. 
\item Remote process injection: create a child process, ``ssh.exe'', that uses loads identified the DLL, then inject a shellcode to initiated
  a remote attack which allowed a back-connect shell session to be opened from a remote system.
\end{enumerate}

An interesting point to note is that the attack code can programmatically construct OS syscall wrappers at run-time without relying
on hard coded definitions or OS data structures.  These clean call stubs are unmonitored, but 
they need to built from a clean and unhooked copy ``NTDLL.DLL''.  The method is 
described in the ``Hell's Gate'' article on virus executable (VX) \autocite{smellyvx:2021}.  The significance of this is that
the functions within the`` NTDLL.dll'' module are wrappers around OS service requests that the malware payload can exploit. 

Also payloads are executed without the creation of a new thread, making the detection of this attack by end-point defences harder.


\subsection{Current Mitigants}

The first line of our defence of an attack based on the Mockingjay process injection technique will be our XDR system to:
\begin{itemize}
\item Consolidation and preparation phase: detect the code injection attack.
\item Target impact phase: detect the actions of the payload execution, which will be determined by the attacker's motives.
\end{itemize}

Using ``Lifecycle of a Ransomware Incident'' produced by \autocite{Certnz:2021} and the modus-operandi of the LockBit
group \autocite{CISA:2023}, we should anticipate:

\begin{enumerate}
\item Direct attack through our application gateway (Internet-exposed Service).
\item Phishing attack through email.
\end{enumerate}

If either of these two approaches are successful, ransomware groups such as LockBit will use the exploit to:

\begin{enumerate}
\item Exfiltrate data.
\item Encrypt data.
\item Destroy backups.
\end{enumerate}


\begin{figure}[ht]
\includegraphics[scale=0.55]{certnz_aa23-165a.png}
\includegraphics[scale=0.6]{certnz_aa23-165b.png}
\caption{Ransomware Layered Mitigations \autocite{Certnz:2021}}
\end{figure}

\pagebreak

%\subsection{Analysis of the impact and consequences of these incidents}
\subsection{Proposed Solutions to Improving Cyber-security Measures}

As Security Joes is a security system vendor, the report itself lists a number of ways that the attack could be detected. In
the first instance, we should be talking to our XDR vendor to ensure we have the functionality to automate these steps.  At
the very least we need to be able to perform ad-hoc scans across our infrastructure and windows systems to flag the following:

\begin{enumerate}
\item Build a custom scan across our real estate and construct a database of trusted PE files that have a default RWX section.
\item Monitor for the launching of GNU utilities, as these were exploited by the attack.
\item Ensure our network scanning is looking for network connections to non-standard ports.
\item Investigate how to detect the loading of NTDLL.DLL without false positives.
\end{enumerate}


\subsection{Recommendations}

Our organisation is committed to the continued review of new attacks and actively researches the possible impact of reported attacks.
This active engagement with the cybersecurity community and our vendors allows us to adapt quickly to changes in the threat landscape.

For this case study, we recommend the following actions under each of the NIST CSF functions:

\begin{itemize}
\item \textbf{Identify}: Ensure all systems are patched for the CVEs identified and engage with our XDR vendor to respond to this new threat.
\item \textbf{Protect}: Review XDR solutions are monitoring systems to detect encryption and exfiltration attempts.
\item \textbf{Detect}: Develop a Mockingjay red-team exercise against our XDR; work with our vendor to track and monitor the loading of specific
  DLLs, including NTDLL.dll, by non-legitimate processes.
\item \textbf{Respond}: Ensure that our organisation can report to all relevant authorities and abide by our legal
  responsibilities for disclosures.
\item \textbf{Recover}: Follow up with internal teams to show recovery procedures from backup are up to date and ensure backups are held for all systems and are immutable. 
\end{itemize}


% Section 4 is an implementation of the attack and will look at manually identifying an infected process and generating artifacts that could be used in automating the process.
\section{So Hard Hard is this Anyway?  Attack!}

Description of implementation, etc

\begin{listing}[!ht]
\inputminted{c++}{tex/code/mockingjay.cpp}
\caption{Mockingjay Implementation C++ code}
\label{listing:1}
\end{listing}

\subsection{Inoculation: Generating artifacts}

Blah....


\pagebreak

\section{Evaluation}


When doing this, the breadth of the topic I would need to summarise became apparent. 

Also, the difficulty of providing cyber-protection also became apparent. 
  

Because of this I pivoted to looking more holistically at the threat landscape and cyber-security teams in an organisation after reading  the HBR report. 

Overall, making this a higher-level discussion on the benefits of resilience over protection made a better topic for this unit, and I am happy with the rewrite and think I have achieved that objective. 

What I have not been so successful in is encapsulating the technical knowledge of Windows process injection and detect methods needed to support the case study and present that in a way that is congruent with the higher-level focus of the overall report. 

\section{Conclusion}

In building a resilient cyber-security organisation, it is not enough to focus on the technical details of the protections that
have been put in place.  The CISO must assume that their organisation will experience a cyber-attack and should be able to discuss
the operational and reputation risks that the organisation will face with the board and senior executives.

We looked at one specific style of attack, process injection, and saw how the evolving set of techniques means that the
organisation's defence is always a shifting game of cat-and-mouse; detection and protection are not always guaranteed.

Looking at this specific new process injection, designed to evade typical end-point defence systems, we built a case study that
also took into account the motivations of a threat actor known to exploit such a vulnerability --- the LockBit malware group ---
to ask how can  use the knowledge of this new technique to strengthen both our defences \textit{and} our planning for our response if ever an
attack is successful.

While the most obvious mitigants to a cyber-event are to engage with vendors on new attack vectors, update and patch existing systems and
restrict the attack surface as much as possible, the CISO must translate these technical responses into business level questions on operational
resilience.

Building board-level trust in cyber-security defences is the foundation on which a CISO can build a successful cyber-security team.  And
building those foundations requires framing the communication of cyber-security issues, not in technical terms, but in the robustness
of the fundamental services that the organisation must deliver every working day.


%Such internal case studies are just one way that a cyber

%The outcome of such reviews should feed into 
%\begin{itemize}
%\item Recap of key points related to malware attacks and process injection on Windows
%\item Call to action for cybersecurity professionals and organizations
%\item Ongoing vigilance and adaptation to the evolving threat landscape
%\item Recap of the challenges identified in the Harvard Business Review article.
%\item Summary of the proposed strategic enhancement for CISOs.
%\item Emphasis on the importance of elevating the CISO's role for improved cybersecurity.
%\end{itemize}

%Communicating cyber and information security risks and mitigants to senior executives and board members is an important primary objective of CISOs.
%By regularly giving honest assesments of their organisations defences and weaknesses, the CISO can improve trust, explain budget requirements, and work to a convergence of opinion within the
%leadership and givernance teams..

%Part of this communication should be regular reporting developed by the information security group that can be used to generate board presentations and executive summaries for reporting on the
%organisation's preparedness.

%The reports deliverss an example an on in-depth review of such an immergent threat; process injection techniques, EDR capabilities, and weaknesses in that defence.  The outcome of that review are
%a number risks with-in the cyber-security system and  steps that the information security group can perform to mitigate those risks.

%The identified risks are:
%\begin{itemize}
%\item Attacks can be from both zero-day exploits of systems, such as secure remote access gateways such as Netscaler as well as delivery through email phishing attacks.
%\item The capabilities of EDRs from different vendors are diffent and may have false negatives for different types of attacks.
%\item \ldots \textit{add in good cyber-security policies}
%\end{itemize}

%Mitigants that the CISO can put into place are:
%\begin{itemize}
%\item Ensure that the organisation has a development plan for security professionals within the orgainisation, and that these people develop skills to test and appraise systems \ldots \textit{red team blue teams, pen test, etc}.
%\item  All systems, not just front line defences, such as anti-virus and EDRs, should have procedures for timely and robust patching and upgrades.
%\item The CISO should ensure that vendors are challenged to  validate their systems for new attacks and regularly test systems capabilities. 
%\end{itemize}

\pagebreak

\printbibliography

%\pagebreak

%\appendix

%\section*{Appendix: Generating Payloads}

\subsection*{Injection Techniques}

\href{https://cocomelonc.github.io/tutorial/2021/09/18/malware-injection-1.html}{malware injectoin}


\href{https://cocomelonc.github.io/tutorial/2021/09/04/simple-malware-av-evasion.html}{simple malware evasion}

\subsection*{Generating shellcode using Metasploit}

\href{https://stackoverflow.com/questions/42289112/generate-shellcode-by-using-msfvenom}{generating shellcode by using msfvenom}



%\section*{Appendix: Terminology and Abbreviations}

\begin{multicols}{2}

  
  \entry{ASLR}{Address space layout randomization is a technique to guard against buffer-overflow attacks.}
  \entry{ICE}{In-memory Code Execution attacks only execute malicious operations in memory and leaving little evidence on disk.}
  \entry{SIEM}{Security information and event management: combining security information management (SIM) and event mangement (SEM), these systems enable organizations detect, analyze, and respond to security threats.  This is achieved by collecting event log data from many sources to identify anomolous activities deviating from the norm with real-time analysis.  These systems seek to take appropriate action before any threat harms business operations.
    security threats before they harm business operations.}
  \entry{SOC}{Security operations center.}

\end{multicols}

\end{document}

%%% Local Variables:
%%% mode: latex
%%% TeX-master: t
%%% End:
